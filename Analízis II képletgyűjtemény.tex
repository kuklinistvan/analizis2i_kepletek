\documentclass[a4paper,12pt]{article}
\usepackage[cm]{fullpage}
\usepackage{graphicx}
\usepackage{amsmath}
\usepackage{amssymb}
\usepackage[hidelinks]{hyperref}
\usepackage{bm}
\usepackage{amsmath}

\usepackage{t1enc}
\usepackage[utf8]{inputenc}
\usepackage[hungarian]{babel}
\usepackage{lmodern}

\usepackage{enumitem}
\setitemize{itemsep=0pt}

\usepackage[dvipsnames]{xcolor}

\begin{document}

\renewcommand{\arraystretch}{1.2}

\definecolor{darkgreen}{HTML}{098905}

\newcommand{\n}{\ensuremath{\textcolor{blue}{n}}}
\renewcommand{\k}{\ensuremath{\textcolor{PineGreen}{k}}}
\newcommand{\x}{\ensuremath{\textcolor{red}{x}}}
\renewcommand{\u}{\ensuremath{\textcolor{darkgreen}{u}}}
\renewcommand{\c}{\ensuremath{\textcolor{Sepia}{c}}}
\newcommand{\q}{\ensuremath{\textcolor{SkyBlue}{q}}}

\newcommand{\sumi}{\ensuremath{\sum_{\n= 0}^{\infty}}}

\title{Analízis 2 informatikusoknak képletgyűjtemény II. zh-ra}
\author{Kuklin István}

\maketitle

\noindent Tároló: \url{https://github.com/kuklinistvan/analizis2i_kepletek/}

\tableofcontents

\newpage
\boldmath

\section{Előszó}
Ezt a képletgyűjteményt a deriválttáblázattal ellentétben nem lehet használni a dolgozatban. Azért állítottam össze, hogy a magolást megkönnyítse. Emellett nem vagyok jó matekból és biztos vagyok benne, hogy tele van hibával ez a dokumentum. Szóval minden pull request-nek és issue-nak örülök.

\section{$\sinh x$ és $\cosh x$}
\begin{minipage}{0.5\textwidth}
\begin{equation}
\sinh \x = \frac{e^{\x} - e^{-\x}}{2}
\end{equation}
\end{minipage}
\begin{minipage}{0.5\textwidth}
\begin{equation}
\cosh \x = \frac{e^{\x} + e^{-\x}}{2}
\end{equation}
\end{minipage}

\section{Sorok, Taylor sor, hatványsorok}
\subsection{Taylor sor általában}

\begin{equation}
\sumi \frac{f^{(\n)}(x_0)}{\n!} \cdot (\x-x_0)^{\n}
\end{equation}

\subsection{Nevezetes Taylor sorok}
\begin{equation}
e^{\x} = \sumi \frac{{\x}^{\n}}{\n!}
\end{equation}
\begin{equation}
\ln \x = \sumi \frac{(-1)^{\n-1}}{\n} \cdot (\x-1)^{\n}
\end{equation}
\begin{equation}
\cos \x = \sumi \frac{(-1)^{\n}}{(2\n)!} \cdot \x^{2\n}
\end{equation}
\begin{equation}
\sin \x = \sumi \frac{(-1)^{\n}}{(2\n+1)!} \cdot \x^{2\n+1}
\end{equation}
\begin{equation}
\cosh \x = \sumi \frac{1}{(2\n)!} \cdot \x^{2\n}
\end{equation}
\begin{equation}
\sinh \x = \sumi \frac{1}{(2\n+1)!} \cdot \x^{2\n+1}
\end{equation}

\subsection{Geometriai / mértani sor}
Hatvány sorba fejtésénél használatos!
\begin{equation}
\sumi \c_1 \cdot \q^{\n} = \frac{\c_1}{1-\q};\ |\q| < 1
\end{equation}

\subsection{Binomiális sor}
\begin{equation}
  (1+\u)^{\k} = \sumi \binom{\k}{\n} \cdot {\u}^{\n}
\end{equation}

\section{Kétváltozós függvények analízise}
\subsection{Hesse mátrix}
\begin{equation}
  \det
\begin{bmatrix}
  f^{''}_{xx}(x_0;y_0) & f^{''}_{xy}(x_0;y_0) \\
  f^{''}_{yx}(x_0;y_0) & f^{''}_{yy}(x_0;y_0)
\end{bmatrix}
= f^{''}_{xx} \cdot f^{''}_{yy} - f^{''}_{xy} \cdot  f^{''}_{yx}
\end{equation}
\begin{itemize}
\item Ha 0, akkor wrecked\footnote{``További vizsgálat szükséges''}.
\item Ha negatív, nyeregpont
\item Ha pozitív:
  \begin{itemize}
  \item Ha $f^{''}_{xx} < 0$: lokális maximum
  \item Ha $f^{''}_{xx} > 0$: lokális minimum
  \end{itemize}
\end{itemize}
\subsection{Érintősík egyenlete}
\begin{equation}
z = f(x_0;y_0) + f^{'}_x(x_0;y_0)(x-x_0) + f^{'}_y(x_0;y_0)(y-y_0)
\end{equation}
\subsubsection{Érintősík normálvektora}
Rendezzük 0-ra a fentit és az együtthatók adják rendre.

\subsection{Totális differenciálhatóság}

\subsubsection{UPDATE: egyszerűbben}
Ha kevés pontért kér indoklást a feladat, akkor a varázsmondat így hangzik: a függvény esetén $\exists \varepsilon > 0$ úgy, hogy $K_{\varepsilon}(x;y)$-ban léteznek $f^ {'}_x(x;y)$ és $f^{'}_y(x;y)$ parciális deriváltak, és azok folytonosak. \\
Ahol kifejezetten nem folytonos a függvény, ott ez természetesen nem igaz.\footnote{Elvileg a mondat igazságtartalma eléggé triviális, de férfiasan elismerem, hogy ötletem sincs, egyáltalán honnan szedtük ezt a megállapítást.}

\subsubsection{Teljeskörűen (nem valószínű, hogy kelleni fog a zh-ban)}

$f(x;y)$ totálisan differenciálható, ha fennáll, hogy
\begin{equation}
  \lim_{(x;y) \rightarrow (x_0;y_0)} \frac{f(x;y) - \overbrace{\left[ f(x_0;y_0) + f^{'}_x(x_0;y_0)(x-x_0) + f^{'}_y(x_0;y_0)(y-y_0)  \right]}^{\text{Érintősík egyenlete}}}{\sqrt{(x-x_0)^2 + (y-y_0)^2}} = 0
\end{equation}
Ha pedig totálisan differenciálható $\Rightarrow$ parciálisan is differenciálható. \emph{Fordítva nem igaz!!}

\newpage

\subsubsection*{Feladatmegoldás definíció alapján}

$\forall \varepsilon > 0: \exists \delta > 0: P(x;y) \in$ vizsgált pont $\delta$ sugarú környezetének - azaz
\[
  \sqrt{(x-x_0)^2 + (y-y_0)^2} < \delta \text{,}
\]
akkor fennáll, hogy

\begin{equation}
  \left| \frac{f(x;y) - \left[ f(x_0;y_0) + f^{'}_x(x_0;y_0)(x-x_0) + f^{'}_y(x_0;y_0)(y-y_0)  \right]}{\sqrt{(x-x_0)^2 + (y-y_0)^2}} \right| < \varepsilon
\end{equation}
\begin{itemize}
\item Helyettesítsünk $x_0$ és $y_0$ értékeket, majd számoljuk ki a parciális deriváltakat.
\item Vonjunk össze, rendezzünk.
\item \emph{Polinom osztással} hozzuk ki belőle $(x-x_0)$-t és $(y-y_0)$-t és becsüljünk.
  \[
    |x-x_0| < \delta;\ |y-y_0| < \delta
  \]
\end{itemize}

\section{Iránymenti derivált}
\subsection{Keresés definícióval}
\begin{equation}
  \frac{\delta f(P_0)}{\delta \vec{v}} = \lim_{t \rightarrow 0} \frac{f(P_0 + t \cdot \vec{v_e}) - f(P_0)}{t}\ ,
\end{equation}
ahol:
\begin{itemize}
\item $P_0$ a vizsgált pont
\item $\vec{v_e}$ annak az iránynak az egységnyi hosszú vektora, amiben a meredekséget keressük
\end{itemize}
\subsection{Iránymenti deriválás gradiensvektorral}
\begin{equation}
  \frac{\delta f}{\delta \vec{v}} = \overbrace{
    \begin{bmatrix}
      f^{'}_x(x_0;y_0) \\
      f^{'}_y(x_0;y_0) \\
      \vdots
    \end{bmatrix}}^{grad\ f} \cdot \vec{v}_e \ ,
\end{equation}
ahol
\begin{itemize}
\item $\vec{v}_e$ az egységnyi hosszúra rövidített irányvektor
\end{itemize}
\subsection{Fontos infó maximális és minimális meredekség keresése}
A gradiensvektor mindig a maximális meredekség irányába néz.


\end{document}
