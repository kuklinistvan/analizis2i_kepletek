\documentclass[a4paper,12pt]{article}
\usepackage[cm]{fullpage}
\usepackage{graphicx}
\usepackage{amsmath}
\usepackage{amssymb}
\usepackage[hidelinks]{hyperref}
\usepackage{bm}

\usepackage{t1enc}
\usepackage[utf8]{inputenc}
\usepackage[hungarian]{babel}
\usepackage{lmodern}

\usepackage{enumitem}
\setitemize{itemsep=0pt}

\usepackage[dvipsnames]{xcolor}

\begin{document}

\renewcommand{\arraystretch}{1.2}

\definecolor{darkgreen}{HTML}{098905}

\newcommand{\n}{\ensuremath{\textcolor{blue}{n}}}
\renewcommand{\k}{\ensuremath{\textcolor{PineGreen}{k}}}
\newcommand{\x}{\ensuremath{\textcolor{red}{x}}}
\renewcommand{\u}{\ensuremath{\textcolor{darkgreen}{u}}}
\renewcommand{\c}{\ensuremath{\textcolor{Sepia}{c}}}
\newcommand{\q}{\ensuremath{\textcolor{SkyBlue}{q}}}

\newcommand{\sumi}{\ensuremath{\sum_{\n= 0}^{\infty}}}

\title{Analízis 2 informatikusoknak képletgyűjtemény II. zh-ra}
\author{Kuklin István}

\maketitle

\tableofcontents

\newpage
\boldmath

\section{Előszó}
Ezt a képletgyűjteményt a deriválttáblázattal ellentétben nem lehet használni a dolgozatban. Azért állítottam össze, hogy a magolást megkönnyítse.

\section{$\sinh x$ és $\cosh x$}
\begin{minipage}{0.5\textwidth}
\begin{equation}
\sinh \x = \frac{e^{\x} - e^{-\x}}{2}
\end{equation}
\end{minipage}
\begin{minipage}{0.5\textwidth}
\begin{equation}
\cosh \x = \frac{e^{\x} + e^{-\x}}{2}
\end{equation}
\end{minipage}

\section{Sorok, Taylor sor, hatványsorok}

\subsection{Nevezetes Taylor sorok}
\begin{equation}
e^{\x} = \sumi \frac{{\x}^{\n}}{\n!}
\end{equation}
\begin{equation}
\ln \x = \sumi \frac{(-1)^{\n-1}}{\n} \cdot (\x-1)^{\n}
\end{equation}
\begin{equation}
\cos \x = \sumi \frac{(-1)^{\n}}{(2\n)!} \cdot \x^{2\n}
\end{equation}
\begin{equation}
\sin \x = \sumi \frac{(-1)^{\n}}{(2\n+1)!} \cdot \x^{2\n+1}
\end{equation}
\begin{equation}
\cosh \x = \sumi \frac{1}{(2\n)!} \cdot \x^{2\n}
\end{equation}
\begin{equation}
\sinh \x = \sumi \frac{1}{(2\n+1)!} \cdot \x^{2\n+1}
\end{equation}

\subsection{Geometriai / mértani sor}
\begin{equation}
\sumi \c_1 \cdot \q^{\n} = \frac{\c_1}{1-\q};\ |\q| < 1
\end{equation}

\subsection{Binomiális sor}
\begin{equation}
  (1+\u)^{\k} = \sumi \binom{\k}{\n} \cdot {\u}^{\n}
\end{equation}

\end{document}
